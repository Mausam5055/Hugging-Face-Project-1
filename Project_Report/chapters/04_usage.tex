\chapter{Usage Manual}

This chapter guides you through running the application and understanding the user interface.

\section{Running the Application}
With your environment active, run the following command in your terminal:
\begin{lstlisting}[language=bash]
python app.py
\end{lstlisting}

\subsection{Startup Logs}
You should see output similar to the following. Note the model loading times:
\begin{verbatim}
Initializing AI Services...
Loading Translation Model: facebook/nllb-200-distilled-600M...
(This may take 10-20 seconds on first run to download 1.2GB)
Loading Summarization Model: facebook/bart-large-cnn...
(This may take 10-15 seconds to download 1.6GB)
Services loaded successfully.
Running on local URL:  http://127.0.0.1:7860
\end{verbatim}

Open your browser (Chrome/Edge/Firefox) and navigate to \url{http://127.0.0.1:7860}.

\section{Interface Overview}
The User Interface (UI) is built with Gradio and consists of three main sections:

\begin{figure}[H]
    \centering
    \begin{tikzpicture}[font=\small]
        % Window frame (Wider)
        \node[draw, thick, rounded corners, minimum width=13cm, minimum height=8cm, fill=white] (window) {};
        
        % Header inside window
        \node[below=0.5cm of window.north, font=\bfseries\Large] {Multilingual Translator \& Summariser};
        
        % Left Column: Input
        \node[draw, fill=hfgrey, minimum width=5cm, minimum height=5cm, below right=3cm and 1cm of window.north west] (input) {};
        \node[above=0.1cm of input, font=\bfseries] {Input Text};
        \node[gray, text width=4cm, align=center] at (input.center) {Paste any language text here...};
        
        % Button
        \node[draw, fill=hfblue, text=white, minimum width=3cm, rounded corners, below=0.3cm of input] {Process};

        % Right Column: Outputs
        % Output 1
        \node[draw, fill=green!5, minimum width=5cm, minimum height=2cm, below left=3cm and 1cm of window.north east] (out1) {};
        \node[above=0.1cm of out1, font=\bfseries] {English Translation};
        \node[gray, text width=4cm, align=center] at (out1.center) {Translated text appears here...};

        % Output 2
        \node[draw, fill=blue!5, minimum width=5cm, minimum height=2cm, below=1cm of out1] (out2) {};
        \node[above=0.1cm of out2, font=\bfseries] {Summary};
        \node[gray, text width=4cm, align=center] at (out2.center) {Summary appears here...};
    \end{tikzpicture}
    \caption{UI Layout Mockup}
\end{figure}

\section{Step-by-Step Workflow}
\begin{enumerate}
    \item \textbf{Input Field:} Copy and paste any text into the large box on the left. The model handles over 200 languages (Spanish, Hindi, French, Japanese, etc.).
    \item \textbf{Process Button:} Click the "Process" button. The button will show a loading spinner while the GPU processes the request.
    \item \textbf{Outputs:}
    \begin{itemize}
        \item \textbf{Top Right (Translation):} The system first translates the input into English. This preserves the full detail of the original text.
        \item \textbf{Bottom Right (Summary):} The system then reads the English translation and creates a 2-3 sentence abstractive summary.
    \end{itemize}
\end{enumerate}

\section{Example Scenarios}

\subsection{Scenario 1: News Article (Hindi)}
\begin{itemize}
    \item \textbf{Input:} "Bharat ek vishaal desh hai..." (A long paragraph about India's geography).
    \item \textbf{Action:} User clicks Process.
    \item \textbf{Translation:} "India is a vast country..."
    \item \textbf{Summary:} "India is geographically diverse with ancient culture."
\end{itemize}

\subsection{Scenario 2: Tech Review (French)}
\begin{itemize}
    \item \textbf{Input:} "Ce nouvel ordinateur portable est incroyablement rapide mais la batterie..."
    \item \textbf{Translation:} "This new laptop is incredibly fast but the battery..."
    \item \textbf{Summary:} "The laptop has excellent performance but poor battery life."
\end{itemize}

\begin{warningbox}
The first request might be slow (10-20 seconds) as the computer "warms up" the models (JIT Compilation). Subsequent requests will be much faster.
\end{warningbox}

\section{Advanced Customization}
Want to change the models? You can modify `src/translation.py`:

\subsection{Using a Smaller Model}
If you are on an older laptop, switch to the 600M distilled version or even the 1.3B version if you have more RAM.
\begin{lstlisting}[language=Python]
# In src/translation.py
self.model_name = "facebook/nllb-200-distilled-600M" # Fast
# self.model_name = "facebook/nllb-200-3.3B"         # Very Slow, More Accurate
\end{lstlisting}

\subsection{Changing the Summary Style}
You can make the summary bullet-pointed by editing `src/summarization.py`:
\begin{lstlisting}[language=Python]
# In src/summarization.py
def summarize(self, text):
    prompt = f"Summarize this in bullet points: {text}"
    # ... rest of code
\end{lstlisting}
