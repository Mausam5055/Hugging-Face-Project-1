\chapter{Project Overview \& Architecture}

\section{Introduction}
In today's interconnected world, language barriers remain a significant challenge. The **Multilingual Translator \& Summariser** is an AI-powered application designed to bridge this gap. It allows users to input text in \textit{any} language, automatically translates it into English, and then provides a concise summary of the content.

\section{Problem \& Solution}
\begin{itemize}
    \item \textbf{The Problem:} Information Overload. Valuable documents (news, research, user feedback) are often available only in local languages and are too long to digest quickly.
    \item \textbf{The Solution:} A single-click pipeline.
    \begin{enumerate}
        \item \textbf{Translate:} Convert local language `X` to English using \texttt{NLLB-200}.
        \item \textbf{Summarize:} Condense the English text using \texttt{BART-Large}.
        \item \textbf{Present:} Display both outputs in a clean, reactive Web UI.
    \end{enumerate}
\end{itemize}

\section{System Design}
The system follows a modular architecture, separating the core AI logic from the user interface.

\begin{figure}[H]
    \centering
    % Resize box ensures it fits exactly within the page margins
    \resizebox{\textwidth}{!}{
    \begin{tikzpicture}[
        node distance=4cm,
        auto,
        block/.style={rectangle, draw=hfblue, thick, fill=white, text width=3.5cm, align=center, rounded corners, minimum height=1.5cm},
        line/.style={draw, very thick, color=darkgrey, ->, >=stealth},
        label/.style={midway, fill=white, font=\footnotesize, text=darkgrey}
    ]
        % Layout: User -> UI -> App (Horizontal)
        \node [block, fill=green!10] (user) {User Input \\ (Any Language)};
        \node [block, right of=user, node distance=5.5cm, fill=hfblue!10] (ui) {Browser Interface};
        \node [block, right of=ui, node distance=5.5cm] (app) {App Controller \\ (\texttt{app.py})};
        
        % Services (Below App)
        \node [block, below of=app, xshift=-3cm, node distance=4cm] (trans) {Translation \\ (\texttt{NLLB-200})};
        \node [block, below of=app, xshift=3cm, node distance=4cm] (summ) {Summarization \\ (\texttt{BART-Large})};

        % Connections
        \draw [line] (user) -- (ui);
        \draw [line] (ui) -- (app);
        
        % App to Services (Orthogonal Paths)
        % App -> Trans
        \draw [line] (app.south) -- ++(0,-0.5) -| node[label, pos=0.25] {1. Raw Text} (trans.north);
        % Trans -> App
        \draw [line] (trans.east) -- ++(0.5,0) |- node[label, pos=0.25] {2. English} (app.east);
        
        % App -> Summ
        % We connect App East to Summ North (via path)
        % Actually, let's go from App South (shifted) to Summ North
        \draw [line] (app.south) ++(1,0) -- ++(0,-0.5) -| node[label, pos=0.25] {3. English} (summ.north);
        
        % Summ -> App
        % Summ East to App East?
        \draw [line] (summ.east) -- ++(0.5,0) |- node[label, pos=0.25] {4. Summary} (app.south);
        
    \end{tikzpicture}
    }
    \caption{Data Flow Architecture}
\end{figure}

\section{Model Decisions}
\begin{description}
    \item[Translation Model:] We selected `facebook/nllb-200-distilled-600M`.
    \begin{itemize}
        \item \textbf{Pros:} Supports 200+ languages, high accuracy, reasonable size (1.2GB).
        \item \textbf{Why Distilled?} The full model is 54GB, which is impossible to run on consumer hardware. The distilled version offers 90\% of the performance at 2\% of the size.
    \end{itemize}
    
    \item[Summarization Model:] We selected `facebook/bart-large-cnn`.
    \begin{itemize}
        \item \textbf{Pros:} State-of-the-art for abstractive summarization on news articles.
        \item \textbf{Why CNN?} It was fine-tuned on the CNN/DailyMail dataset, making it excellent for understanding journalistic and factual content.
    \end{itemize}
\end{description}
